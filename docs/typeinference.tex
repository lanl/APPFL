\documentclass[11pt]{article}

% preamble

\newcommand{\bu}{\vdash_\mathrm{BU}}
\newcommand{\cA}{\mathcal{A}}
\newcommand{\cC}{\mathcal{C}}
\usepackage{bussproofs}
\usepackage{geometry}
\geometry{
  letterpaper,
  lmargin=1in,
  rmargin=1in,
  tmargin=1in,
  bmargin=1in
}
\usepackage{amssymb}
\let\oldemptyset\emptyset
\let\emptyset\varnothing

\begin{document}

% top matter

\title{Bottom-up Type Inference for Realistic Functional Languages\\ \Large{WORKING DRAFT}}
\author{Kei Davis and David Ringo}
%\date{January 2015}
\maketitle

% \renewcommand{\abstractname}{Executive Summary}
\begin{abstract}
  The great majority of papers on Hindley-Milner type inference provide rules
  for only the simplest of type systems, for example lacking sum or product
  types, and the simplest of language constructs, for example lacking mutually
  recursive definitions, sum and product construction and deconstruction, and
  more generally arbitrary algebraic data types, leaving type inference for
  more realistic languages as ``straightforward'' generalizations or
  extensions.  Anecdotal evidence suggests that practitioners in fact find
  generalization or extension less than straightforward.  Indeed, prior to
  Heeren et al.'s landmark work~\cite{HHS02}, which post-dates Jones' notable
  exception to the treatment of simplistic languages and type
  systems~\cite{Jones2000}, obtaining a sound ordering in the various steps
  (specialization, generalization, etc.)  was arguably a black art.  Using the
  \emph{bottom-up} approach of Heeren et al.\ we give type inference rules for
  two languages: the APPFL compiler \cite{us} intermediate language modeled
  after GHC's STG intermediate language~\cite{PJ??}, and a simple, generic,
  higher-order polymorphic functional language designed to facilitate the
  explication and implementation of a family of demand analysis techniques.
  Both use a common sub-language for defining algebraic data types including
  user-defined \emph{unboxed types}, the latter for which typing rules are
  given as suggested by Peyton Jones and Launchbury~\cite{PJL??}.  For
  comparison and as a starting point we first present Heeren et al.'s language
  and inference rules.
\end{abstract}

% main body

\section{Introduction}
We follow the approach of Heeren et al.~\cite{HHS02}.

\section{Language---Heeren, Hage, Swierstra}

Variable, application, lambda abstraction, non-recursive let.

\section{Language---Datatype Declarations}

The language of datatypes is shared among all languages and is given in Figure~\ref{fig:typedecsyntax}.

\setlength{\tabcolsep}{5pt}
\begin{figure}[t]
\centering
\footnotesize % tiny scriptsize footnotesize small
\begin{tabular}{r r c l l}
  Type constr.\ defn. &  $\mathit{tdef}$ & ::= & \texttt{data} [\texttt{unboxed}]\ 
  $T\ \beta_1 \dots \beta_m\ \mbox{\texttt{=}} $ & $m \ge 0,\ n \ge 1$\\
  & & & \quad $C_1\ \tau_{1,1} \dots \tau_{1,a_1}\ \mbox{\texttt{|}}\ \dots\ \mbox{\texttt{|}}\ C_n\ t_{1,1} \dots t_{1,a_n}  $ \\
\\
Type               & $\tau$           & ::= & \emph{Primtype}           & Primitive type\\
                   &                  & $|$ & $\beta$                   & Type variable \\
                   &                  & $|$ & $\tau$ \texttt{->} $\tau$ & Function type \\
                   &                  & $|$ & $T\ \tau\ \dots\ \tau$    & Type constructor\\
\end{tabular}
\caption{Type Declaration Syntax}
\label{fig:typedecsyntax}
\end{figure}

\section{Language---STG}

Our STG language is modeled after Peyton Jones and Marlow's variant of STG as described in the ``fast curry''
paper~\cite{PJM??}.

\setlength{\tabcolsep}{5pt}
\begin{figure}
\centering
\footnotesize % tiny scriptsize footnotesize small
\begin{tabular}{r r c l l}
Variable     & $f,\ x$        &     &                                              & Initial lower-case letter \\
\\
Constructor  & $C$            &     &                                              & Initial upper-case letter \\
\\
Atom         & $a$            & ::= & $i\ |\ x$                                    & Variable or integer literal\\
\\
Expression   & $e$            & ::= & $a$                                          & Atom \\
             &                & $|$ & $f\ a_1\dots a_n$                            & Application, $n\ge 1$ \\
             &                & $|$ & $\oplus\ a_1\dots a_n$                       & Saturated primitive operation, $n\ge 1$ \\

             &                & $|$ & \texttt{let}                                 & Recursive let, $n\ge 1$ \\
             &                &     & \texttt{ } $\mathit{odecl}_1$ \\
             &                &     & \hspace{0.2in} $\dots$ \\
             &                &     & \texttt{ } $\mathit{odecl}_n$ \\
%             &                &     & \texttt{ } $x_1$ \texttt{=} $\mathit{obj}_1$ \\
%             &                &     & \hspace{0.2in} $\dots$ \\
%             &                &     & \texttt{ } $x_n$ \texttt{=} $\mathit{obj}_n$ \\
             &                &     & \texttt{in} $e$  \\
\\
             &                & $|$ & \texttt{case} $e$ \texttt{of} $\mathit{alts}$ & Case expression (as implemented) \\
\\
             &                & $|$ & \texttt{case} $e$ \texttt{as} $x$ \texttt{of} $\mathit{alts}$ & Case expression (proposed) \\
\\
Alternatives & $\mathit{alts}$ & ::= & \texttt{ } $\mathit{alt}_1$                  & Case alternatives, $n \ge 1$\\
             &                &     & \hspace{0.2in} $\dots$ \\
             &                &     & \texttt{ } $\mathit{alt}_n$ \\
\\
Alternative  & $\mathit{alt}$ & ::= & $C\ x_1\dots x_n$ \texttt{->} $e$             & Pattern match, $n \ge 0$ \\
             &                & $|$ & $i$ \texttt{->} $e$                           & Integer literal \\
             &                & $|$ & $x$ \texttt{->} $e$                           & Default (as implemented)\\
             &                & $|$ & \texttt{\char`_\ ->} $e$                      & Default (proposed)\\
\\
Object       & $\mathit{obj}$ & ::= &\texttt{FUN} $f\ x_1\dots x_n$ \texttt{->} $e$ & Function definition, arity $=n\ge 1$ \\
             &                & $|$ &\texttt{CON} $C\ a_1\dots a_n$                 & Saturated constructor, $n \ge 0$ \\
             &                & $|$ &\texttt{THUNK} $e$                             & Thunk---explicit deferred evaluation \\
%             &                & $|$ &\texttt{PAP} $f\ a_1\dots a_n$               & Partial application \\
%             &                & $|$ & $\mathit{BLACKHOLE}$                         & Evaluation-time black hole \\
\\
Object decl. & $\mathit{odecl}$ & ::=  & $x = \mathit{obj}$                         & Simple binding \\
\\
%%%Constructor defn. & $\mathit{con}$  & ::= & $C\ \mathit{type}_i$ & $i \ge 0$ \\
%%%\\
%%%Datatype defn. &  $\mathit{ddecl}$ & ::= & \texttt{data} [\texttt{unboxed}] & User-defined data type  \\
%%%               &                   &     & $C\ x_i =$                       & $i \ge 0, n > 0$         \\
%%%               &                   &     & $\mathit{con}_1 | \dots |  \mathit{con}_n$ \\
%%%\\
%%%Program      & $\mathit{prog}$ & ::= & $\mathit{(o|d)decl}_1$ \texttt{;}            & Object and data defns, \\
%%%             &                 &     & \texttt{ } $\dots$ \texttt{;}                & distinguished \texttt{main}\\
%%%             &                 &     & $\mathit{(o|d)decl}_n$ & 
Program      & $\mathit{prog}$ & ::= & $\mathit{odecl}_1$ \texttt{;}                & Object and data defns, \\
             &                 &     & \texttt{ } $\dots$ \texttt{;}                & distinguished \texttt{main}\\
             &                 &     & $\mathit{odecl}_n$ & 
%Program      & $\mathit{prog}$& ::= & $f_1\ =\ \mathit{obj}_1$ \texttt{;}          & $n \ge 1$, distinguished \texttt{main}\\
%             &                &     & \texttt{ } $\dots$ \texttt{;} \\
%             &                &     & $f_n\ =\ \mathit{obj}_n$

\end{tabular}
\caption{STG syntax}
\label{fig:STGsyntax}
\end{figure}

\clearpage
\section{Type inference rules}
Judgements are of the form $\cA,\ \cC\ \vdash_{\cdot}\ e:\tau$, where $\cdot$ may be empty (the default),
or to facilitate the factoring
of the rule for \texttt{case} one of $x$ (passing the name of the scrutinee-bound variable) or
$\tau$ (passing the type of the scrutinee).   Subscript index ranges
are implicitly universally quantified over the relevant range except where needed for to avoid ambiguity.

Starting with the base cases, each leaf (Expression/Atom) variable is
associated with a fresh type variable, and each literal is associated with its
(unambiguous) type, e.g., $i:\mathit{Int}$ or $i:\mathit{Int\#}$, and
Primitive operators are assumed to be monomorphic.
For recursive \texttt{let} it is convenient to separate the \texttt{let} and \texttt{in} clauses.

TODO:  describe additions to Heeren et al.\ scheme for unboxed data types.

\begin{figure}
  \small
% Variable/Atom
%
\begin{prooftree}
  \AxiomC{}
  \LeftLabel{[Variable/Atom]}
  \UnaryInfC{$\{x:\beta\},\ \emptyset\ \vdash\ x:\beta\ (\beta\ \mathrm{fresh})$}
\end{prooftree}
%
% Literal/Atom
%
\begin{prooftree}
  \AxiomC{}
  \LeftLabel{[Literal/Atom]\quad}
  \UnaryInfC{$\emptyset,\ \emptyset\ \vdash\ i:\mathrm{Int}$}
\end{prooftree}
%
% Expression/Application
%
\begin{prooftree}
  \AxiomC{$\cA_i,\ \cC_i\ (=\emptyset)\ \vdash\ a_i:\tau_i$}
  \LeftLabel{[Expression/Application]\quad}
  \UnaryInfC{$\bigcup \cA_i \cup \{f:\tau_1 \rightarrow \ldots \rightarrow \tau_n \rightarrow \beta \}
      \ \vdash f\ a_1 \ldots a_n : \beta\ (\beta\ \mathrm{fresh})$ }
\end{prooftree}
%
% Primitive Operation
%
\begin{prooftree}
  \AxiomC{$\cA_i,\ \cC_i\ \vdash\ a_i:\tau_i$}
  \AxiomC{$\oplus : \tau_1 \rightarrow \ldots \rightarrow \tau_n \rightarrow \tau_{n+1}$}
  \LeftLabel{[Primitive Operation]\quad}
  \BinaryInfC{$\bigcup \cA_i,\ \{ \tau_i \equiv \tau_i \}\ \vdash\ \oplus\ a_1 \ldots a_n : \tau_{n+1}$}
\end{prooftree}
%
% let
%
\begin{prooftree}
  \AxiomC{$\cA_i,\ \cC_i\ \vdash\ e_i:\tau_i,\ 1\le i \le n$}
  \LeftLabel{[\texttt{let}]\quad}
  \UnaryInfC{$(\bigcup \cA_i)\backslash \{x_i\},\ \bigcup \cC_i \cup \{\tau' \equiv \tau_i\ |\ x_i : \tau' \in \cA_j,\
    1 \le i \le n,\ 1 \le j \le n \}\ \vdash\ \mbox{\texttt{let}}\ x_i = e_i : \tau_i$}
\end{prooftree}
%
% in
%
\begin{prooftree}
  \AxiomC{$\cA,\ \cC\ \vdash\ \mbox{\texttt{let}}\ x_i = e_i : \tau_i$}
  \AxiomC{$\cA_0,\ \cC_0\ \vdash\ e_0:\tau_0$}
  \LeftLabel{[\texttt{in}]\quad}
  \BinaryInfC{$\cA \cup (\cA_0 \backslash \{x_i\}),\ \cC \cup \cC_0 \cup
    \{ \tau' \le_{M} \tau_i\ |\ x_i:\tau' \in \cA_0 \}
    \ \vdash\ \mbox{\texttt{let}}\ x_i = e_i\ \mbox{\texttt{in}}\ e_0:\tau_0$}
\end{prooftree}
%
% case
%
\begin{prooftree}
  \AxiomC{$\cA_0,\ \cC_0\ \vdash\ e_0 : \tau_0$}
  \AxiomC{$\cA_\mathit{alts},\ \cC_\mathit{alts}\ \vdash_{\tau_0}\ \mathit{alts} : \tau_\mathit{alts}$}
  \LeftLabel{[\texttt{case}]\quad}
  \BinaryInfC{$\cA_0 \cup \cA_\mathit{alts},\ 
    \cC_0 \cup \cC_\mathit{alts} \vdash\
    \mbox{\texttt{case}}\ e_0\ \mbox{\texttt{of}}\ \mathit{alts}:\tau_\mathit{alts}$}
\end{prooftree}
%
% alts
%
\begin{prooftree}
  \AxiomC{$\cA_i,\ \cC_i\ \vdash_{\tau_0}\ \mathit{alt}_i : \tau_i$}
  \LeftLabel{[\texttt{alts}]\quad}
  \UnaryInfC{$\bigcup \cA_i,\  \bigcup \cC_i \cup \{ \tau_1 \equiv \tau'\ |\ 2 \le i \le n \}\ \vdash_{\tau_0}\
    \mathit{alts} : \tau_1 $}
\end{prooftree}
%
% alt/constructor
%
%
% alt/integer literal
%
%
% alt/default var
%
%
% alt/default anon
%
%
% FUN
%
\begin{prooftree}
  \AxiomC{$\cA,\ \cC\ \vdash\ e : \tau$}
  \LeftLabel{[FUN]\quad}
  \UnaryInfC{$\cA \backslash \{x_i\} \cup \{f : \beta_1 \rightarrow \dots \rightarrow \beta_n \rightarrow \tau\} \ (\beta_i\ \mathrm{fresh}),$}
  \noLine
  \UnaryInfC{$\cC \cup \{\tau' \equiv \beta_i\ |\ x_i : \tau' \in \cA,\ 1\le i \le n\}$}
  \noLine
  \UnaryInfC{$\vdash\ \mbox{\texttt{FUN}}\ f\ x_1 \dots x_n\ \mbox{\texttt{->}}\ e : \beta_1 \rightarrow \dots \rightarrow \beta_n \rightarrow \tau$}
\end{prooftree}
%
% CON
%
\begin{prooftree}
  \AxiomC{$\cA_i,\ \cC_i\ (=\emptyset)\ \vdash\ a_i : \tau_i$}
  \AxiomC{$T\ \beta_1 \dots \beta_j \rightarrow C\ m_1 \dots m_n\ (\beta_i\ \mbox{fresh})$}
  \LeftLabel{[CON]\quad}
  \BinaryInfC{$\bigcup \cA_i,\ \{\tau_i \equiv m_i\}\ \vdash\ \mbox{\texttt{CON}}\ C\ a_1 \dots a_n : T\ \beta_1 \dots \beta_j$}
\end{prooftree}
%
% THUNK
%
\begin{prooftree}
  \AxiomC{$\cA,\ \cC\ \vdash\ e : \tau$}
  \LeftLabel{[THUNK]\quad}
  \UnaryInfC{$\cA,\ \cC\ \vdash\ \mbox{\texttt{THUNK}}\ e : \tau$}
\end{prooftree}
\caption{STG Bottom-up Type Inference Rules (as implemented)}
\label{fig:BUSTG}
\end{figure}

\begin{figure}
  \small
%
% case
%
\begin{prooftree}
  \AxiomC{$\cA_i,\ \cC_i\ \vdash\ \mathit{alt}_i:\tau_0$}
  \AxiomC{$\cA_\mathit{alts},\ \cC_\mathit{alts}\ \vdash_{x}\ \mathit{alts} : \tau_\mathit{alts}$}
  \LeftLabel{[\texttt{case}]\quad}
  \BinaryInfC{$\cA_0 \cup (\cA_\mathit{alts}\backslash\{x\}),\ 
    \cC_0 \cup \cC_\mathit{alts} \cup \{\tau_0 \equiv \tau'\ |\ x:\tau' \in \cA_\mathit{alts} \}\ \vdash\
    \mbox{\texttt{case}}\ e_0\ \mbox{\texttt{as}}\ x\ \mbox{\texttt{of}}\ \mathit{alts}:\tau_\mathit{alts}$}
\end{prooftree}
%
% alts
%
\begin{prooftree}
  \AxiomC{$\cA_i,\ \cC_i\ \vdash_{x}\ \mathit{alt}_i : \tau_i$}
  \LeftLabel{[\texttt{alts}]\quad}
  \UnaryInfC{$\bigcup \cA_i,\  \bigcup \cC_i \cup \{ \tau_1 \equiv \tau_i\ |\ 2 \le i \le n \}\ \vdash\
    \mathit{alts} : \tau_1 $}
\end{prooftree}
%
% alt/constructor
%
\begin{prooftree}
  \AxiomC{$\cA,\ \cC\ \vdash\ e : \tau$}
  \AxiomC{$T\ \beta_1 \dots \beta_j \rightarrow C\ \tau_1 \dots \tau_n\ (\beta_i\ \mbox{fresh})$}
  \LeftLabel{[\texttt{alt constructor}]\quad}
  \BinaryInfC{$\cA\backslash\{x_i\} \cup \{ x : T\ \beta_1 \dots \beta_j \},\
    \cC \cup \{\tau_i \equiv \tau'\ |\ x_i : \tau' \in \cA \}\ \vdash\ C\ x_1 \dots x_n\ \mbox{\texttt{->}}\ e : \tau$}
\end{prooftree}
%
% alt/integer literal
%
\begin{prooftree}
  \AxiomC{$\cA,\ \cC\ \vdash\ e : \tau$}
  \LeftLabel{[\texttt{alt literal int}]\quad}
  \UnaryInfC{$\cA \cup \{x : \mathit{Int}\},\ \cC\ \vdash\ i\ \mbox{\texttt{->}}\ e : \tau$}
\end{prooftree}
%
% alt/default var
%
%
% alt/default anon
%
\begin{prooftree}
  \AxiomC{$\cA,\ \cC\ \vdash\ e : \tau$}
  \LeftLabel{[\texttt{alt anon}]\quad}
  \UnaryInfC{$\cA,\ \cC\ \vdash\ \mbox{\texttt{\char`_\ ->}}\ e : \tau$}
\end{prooftree}
%
\caption{STG Proposed Inference Rules for \texttt{case}}
\label{fig:proposed}
\end{figure}


\section{Language---Realistic Higher-order, Polymorphic, Pure Functional}

David's stuff goes here.

  
\end{document}

Using the command \EnableBpAbbreviations enables some laconic shorthand for various commands:
\AX and \AXC abbreviate \Axiom and \AxiomC
\UI and \UIC abbreviate \UnaryInf and \UnaryInfC
\BI and \BIC abbreviate \BinaryInf and \BinaryInfC
\TI and \TIC abbreviate \TrinaryInf and \TrinaryInfC
\DP abbreviates \DisplayProof
