% This is LLNCS.DEM the demonstration file of
% the LaTeX macro package from Springer-Verlag
% for Lecture Notes in Computer Science,
% version 2.4 for LaTeX2e as of 16. April 2010
%
\documentclass{llncs}
%
\usepackage{makeidx}  % allows for indexgeneration
%
\begin{document}
%
%mkd \frontmatter          % for the preliminaries
%
\pagestyle{headings}  % switches on printing of running heads
%
%mkd \mainmatter              % start of the contributions
%
\title{Strict Pure Functional Programming and Automatic Parallelization}
%
\titlerunning{Automatic Parallelization}  % abbreviated title (for running head)
%                                     also used for the TOC unless
%                                     \toctitle is used
%
\author{Kei Davis\inst{1} \and Dean Prichard\inst{1} \and David Ringo\inst{1,2}}
%
\authorrunning{Kei Davis et al.} % abbreviated author list (for running head)
%
%%%% list of authors for the TOC (use if author list has to be modified)
%\tocauthor{Ivar Ekeland, Roger Temam, Jeffrey Dean, David Grove,
%Craig Chambers, Kim B. Bruce, and Elisa Bertino}
%
\institute{Los Alamos National Laboratory, Los Alamos, NM, USA\\
\email{kei.davis@lanl.gov},\\ WWW home page:
\texttt{http://ccsweb.lanl.gov/\homedir kei/}
\and
University of New Mexico,
Albuquerque, NM, USA}

\maketitle              % typeset the title of the contribution

\begin{abstract}
The trend is functional programming in scientific computing, and in particular
strict (by default) functional programming, in various guises.  The project is
the demonstration of a light-weight, higher-order, polymorphic,
strict-by-default purely functional language implementation in which we can
experiment with automatic parallelization strategies and varying degrees of
default strictess.

\keywords{strict pure functional programming, automatic parallelization}

\end{abstract}

\section{Background and Motivation}

\section{Goals}

\section{Design and Implementation}

\section{Current Status and Expected Results (by the time of the symposium)}


%
% ---- Bibliography ----
%
\begin{thebibliography}{5}

\end{thebibliography}


\end{document}
